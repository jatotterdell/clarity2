\documentclass[11pt,parskip=half-]{scrartcl}
\usepackage[headsepline,footsepline]{scrlayer-scrpage}
\usepackage{amsmath, amssymb}
\usepackage{fontspec}
\usepackage{newpxtext,newpxmath}
\usepackage{graphicx,grffile}
\usepackage{setspace}
\usepackage{booktabs}
\usepackage{hyperref}
\usepackage{float}
\usepackage{enumitem}
\usepackage{lastpage}
\usepackage[backend=biber,style=nature]{biblatex}

\addbibresource{clarity2_sap.bib} %Import the bibliography file

\renewcommand\pagemark{{\usekomafont{pagenumber}\thepage\ of \pageref{LastPage}}}

\hypersetup{
    colorlinks=true,
    linkcolor=blue,
    filecolor=magenta,      
    urlcolor=cyan,
    pdftitle={CLARITY 2 -  Statistical Analysis Plan},
    pdfpagemode=FullScreen,
    }

% Setup header
\automark*{section}
\clearpairofpagestyles
\pagestyle{scrheadings}
\ihead{\headmark}
\ifoot{CLARITY 2.0 SAP - Version 0.1}
\ofoot{\pagemark}

% Setup line spacing
\renewcommand{\arraystretch}{1}
\onehalfspacing
\AfterTOCHead{\singlespacing}

%\RedeclareSectionCommand[
%  afterskip=1sp]{section}
%\RedeclareSectionCommand[
%  afterskip=1sp]{subsection}
%\RedeclareSectionCommand[
%  afterskip=1sp]{subsubsection}

% Custom commands
\providecommand{\tightlist}{%
  \setlength{\itemsep}{0pt}\setlength{\parskip}{0pt}}

  % enumitem options
\setlist[itemize]{parsep=2pt}
\setlist[enumerate]{parsep=2pt}


\begin{document}
\title{Statistical Analysis Plan}
\subtitle{CLARITY 2}
\author{James Totterdell}
\date{\today}
\makeatletter
\begin{titlepage}
    \begin{center}
        \includegraphics[width=0.7\linewidth]{clarity-logo.jpg}\\[4ex]
        {\huge \bfseries  \@title }\\[2ex]
        {\LARGE \bfseries  \@subtitle }\\[2ex]
        {\large Version 0.1}\\[2ex]
        {\large \@date}\\[10ex]
        \fbox{\begin{tabular}{p{0.20\linewidth}p{0.5\linewidth}}
                Sponsor:         & The University of Sydney \\
                Registration:    & CTC0312                  \\
                Study Co-Chairs: & Professor Meg Jardine    \\
                                 & Professor Vivekanand Jha
            \end{tabular}}
    \end{center}
\end{titlepage}
\makeatother
\thispagestyle{empty}
\newpage

\tableofcontents

\clearpage

\section*{Preface}

This statistical analysis plan (SAP) outlines the data and procedures for analysing effectiveness and safety of trial interventions from the protocol CLARITY 2.0: An Investigator Initiated, International Multi-Centre, Multi-Arm, Multi-Stage Randomised Double Blind Placebo Controlled Trial of Angiotensin Receptor Blocker (ARB) \& Chemokine Receptor Type 2 (CCR2) Antagonist for the Treatment of COVID-19

The following documents were reviewed when preparing this SAP:
\begin{itemize}
    \item CLARITY 2.0 Study Protocol version 2.0 24 September 2021
\end{itemize}

The planned analyses are similar to those undertaken in the CLARITY trial \cite{hockham2021protocol, mcgree2021controlled}.

\section*{Version History}

\begin{table}[H]
    \begin{center}
        \begin{tabular}{lllp{5cm}}
            \hline
            Version & Date       & Author           & Description \\
            \hline
            0.1     & 2022-02-08 & James Totterdell & First draft \\
            \hline
        \end{tabular}
    \end{center}
\end{table}

\section*{Abbreviations}

\clearpage

\section{Trial Objectives}

\subsection{Primary Objective and Outcome}
The primary objective is to evaluate the safety and efficacy of dual treatment with repagermanium and candesartan in patients hospitalised with COVID-19 disease, assessed by: \textbf{Clinical Health Score at day 14.}

This Clinical Health Score is determined within an 8-point ordinal scale of health status which is a modified version of the 9-point score developed by the WHO for Coronavirus Disease 2019 (COVID-19) trials. A single score will be reported with higher values corresponding to worse symptoms. The ordinal scale is an assessment of the clinical status of the participant at the first assessment for the day, measured at day 14 after the date of randomisation.

The 8-point ordinal scale used for Clinical Health Score is:
\begin{enumerate}[nolistsep]
    \item Not hospitalised, no limitations on activities.
    \item Not hospitalised, limitation on activities.
    \item Hospitalised, not requiring supplemental oxygen.
    \item Hospitalised, requiring supplemental oxygen by mask or nasal prongs.
    \item  Hospitalised, on non-invasive ventilation or high-flow oxygen devices.
    \item Hospitalised, requiring intubation and mechanical ventilation.
    \item Hospitalised, on invasive mechanical ventilation and additional organ support (ECMO).
    \item Death.
\end{enumerate}

\subsection{Secondary Objectives and Outcomes}
The secondary objectives are to evaluate the safety and efficacy of dual treatment with repagermanium and candesartan in patients hospitalised with COVID-19 disease, assessed by:
\begin{enumerate}
    \item Clinical Health Score at day 28.
    \item ICU admission (incidence in days 0-28).
    \item Death (incidence in days 0-28).
    \item Time to death, assessed from hospital admission to death.
    \item Acute Kidney Injury (incidence in days 0-28).
    \item Respiratory Failure (incidence in days 0-28).
    \item Length of hospital admission (days of inpatient stay from admission to discharge or death).
    \item Length of ICU Admission (days in ICU from admission to transfer to ward or death).
    \item Requirement of ventilatory support (count of days with ventilation in days 0-28).
    \item Requirement of dialysis (count of days with dialysis in days 0-28).
    \item Clinical Health Score at day 60.
    \item Clinical Health Score at day 90.
    \item Clinical Health Score at day 180.
\end{enumerate}

\subsection{Safety Objectives and Outcomes}
The specific safety objectives are to evaluate the safety of dual treatment with repagermanium and candesartan in patients hospitalised with COVID-19 disease, assessed by incidence of pre-specified clinical events:

\begin{enumerate}[resume]
    \item Hypotension, requiring an urgent or non-urgent intervention (e.g., reduction in dose or cessation of anti-hypertensive, vasopressors, intravenous fluids). Incidence in days 0-28.
    \item  Hyperkalaemia (defined as a K>5.5-6.0 mmol/L or requiring an intervention including hospitalisation; K>6.0 mmol/L). Incidence in days 0-28.
    \item Deranged Liver Function Tests (defined as Aspartate Aminotransferase (AST) and Alanine Aminotransferase (ALT) >Upper Limit of Normal (ULN) or >1.5 times baseline). Incidence in days 0- 28.
    \item Total SAEs
\end{enumerate}

\clearpage

\section{Study Design}
CLARITY 2.0 is a prospective, Multi-Centre, Multi-Arm Multi-Stage Randomised, Double Blind, Placebo Controlled Trial, utilising adaptive sample size re-estimation. Stage 1 will include 80 patients for a Phase II safety analysis to be conducted in India. Stage 2 will include 520 participants for review of preliminary efficacy data. Expansion to Stage 3, a full Phase III study, will be subject to review of accumulated data in Stage 1 and Stage 2.

Protocol Stage 1 will be conducted in India only. The rest of the world will initiate the study in Stage 2 regardless of the recruitment status of the Stage 1. Recruitment in India will not continue to Stage 2 until completion of the Stage 1 safety analysis and review and approval of the Indian Central Drugs Standard Control Organization Subject Expert Committee on COVID-19 related proposals.


\subsection{Target Population}
Participants must meet all the inclusion criteria, and none of the exclusion criteria, to be eligible for this trial. No exceptions will be made to these eligibility requirements at the time of randomisation. All enquiries about eligibility should be addressed by contacting the sponsor prior to randomisation.

Adults with laboratory-confirmed diagnosis of SARS-CoV-2 infection intended for hospital admission for management of COVID-19.

\subsubsection{Eligibility}
\textbf{Inclusion criteria:}
\begin{itemize}
    \item Adults aged between 18 and 65 years.
    \item Laboratory-confirmed diagnosis of SARS-CoV-2 infection within 10 days prior to randomisation. (Confirmation must be through Reverse Transcription Polymerase Chain Reaction [RT-PCR] method)
    \item Intended for hospital admission for management of COVID-19.
    \item  Patients with moderate (respiratory rate of $\geq$ 24/minute or SPO2: 90\% to $\leq$ 93\% on room air) or severe (respiratory rate of $\geq$ 30/minute or SPO2: <90\% on room air) COVID-19.
    \item  Systolic Blood Pressure (SBP) $\geq$ 120 mmHg OR SBP $\geq$ 115 mmHg and currently treated with a non-RAASi BP lowering agent that can be ceased.
    \item Willing and able to comply with all study requirements, including treatment, timing and/or nature of required assessments.
    \item Documented informed consent
\end{itemize}

\textbf{Exclusion criteria:}
\begin{itemize}
    \item Currently treated with an ACEi, ARB or aldosterone antagonist, aliskiren, or ARNi
    \item Intolerance of ARBs
    \item Serum potassium >5.5 mmol/L
    \item An estimated Glomerular Filtration Rate (eGFR) <30ml/min/1.732m
    \item Known biliary obstruction, known severe hepatic impairment (a Child-Pugh-Turcotte score 10-15)
    \item Pregnancy, lactation, or inadequate contraception.
          \begin{itemize}
              \item Female participants must be either post-menopausal, infertile or use a reliable means of contraception for during the treatment period and for at least 60 days after the last dose of investigational product or refrain. Where they are of childbearing potential, female participants must also have a negative pregnancy test result within 7 days prior to randomisation.
              \item Male participants must have been surgically sterilised or use a (double if required) barrier method of contraception during the treatment period and for at least 60 days after the last dose of investigational product or refrain from donating sperm during this period.
          \end{itemize}
    \item Participation in a study of a novel investigational product within 28 days prior to randomisation.
    \item Plans to participate in another study of a novel investigational product during this study.
\end{itemize}

\subsection{Interventions}
\textbf{Investigational arm (C+R):} Titratable candesartan with commencing dose 4mg tablets twice daily (daily dose 8 mg) + fixed dose repagermanium one x 120mg immediate release capsule twice daily (total daily dose 240mg).

\textbf{Control arm \#1 (C+P):} Titratable candesartan with commencing dose 4mg tablets twice daily (daily dose 8 mg) + matched placebo repagermanium one capsule twice daily.

\textbf{Control arm \#2 (P+P):}  Titratable matched placebo candesartan
one tablet twice daily + matched placebo repagermanium one capsule twice daily.

Treatment for will continue for 28 days.

\subsection{Randomisation}
For stage 1, treatment allocation will be 1:1 randomisation via block randomisation stratified by centre between the investigational arm (C+R) and control arm 1 (C+P).

Following stage 1, in stage 2 the three arms will be randomised 1:1:1 using block randomisation stratified by centre.

\subsection{Blinding}
The packaging and labelling of interventions will be designed to maintain blinding to the study team and to participants (double-blind).

\subsection{Sample Size}
Stage 1 of the trial will recruit 80 participants from India for a safety analysis.

Stage 2 of the trial will recruit an additional 520 participants. Information from other relevant trials will be utilised to inform the decision for dropping one of the two control arms.

Progression to stage 3 will be subject to review of the combined stage 1 and stage 2 data.

\clearpage

\section{Descriptive Analyses}

\subsection{Participant Throughput}
The flow of participants through the trial will be summarised for each arm using a CONSORT diagram. The flow diagrams will describe the numbers of participants randomly allocated, who received allocation, withdrew consent, and included in the ITT analysis population.

\subsection{Baseline Characteristics}
The following characteristics will be described separately for patients randomised to each arm:
\begin{itemize}
    \item age at randomisation
    \item sex
    \item ethnicity
    \item weight
    \item height
    \item smoking history
    \item comorbidities
    \item concomitant medications
    \item recent blood test results
\end{itemize}
In general, discrete data will be summarised by counts and proportions. Continuous variables will be summarised by median, lower quartile, upper quartile, minimum and maximum, or where appropriate, by mean and standard deviation. Counts and proportions of missing baseline values will be reported.

\subsection{Completeness of Follow-up}
The number and percentage of participants with follow-up information at day 14 and at day 28 post randomisation will be reported. Patterns of missingness will be summarised for primary and secondary outcomes.

\subsection{Treatment Adherence}
The number and proportion of patients who did not receive the treatment they were allocated to will be reported (if any). Concomitant medications received during the treatment period will be reported. Details on the number of days/doses of treatment received will be reported for each arm and compared with the treatment protocol. The number and proportion of participants who discontinued treatment, the timing of discontinuation, and the reasons for discontinuation as specified in the protocol will be reported.

\clearpage

\section{Comparative Analyses}

Pairwise comparisons will be made between the investigational arm and each control arm. For all outcomes, the primary analyses will be performed with treatment groups as assigned by randomisation (de facto estimand) irrespective of treatment withdrawal or failure to comply with the protocol.

For all models, any baseline covariates included for adjustment will be mean-centred where appropriate, and the treatment design (denoted by $x$) will use orthonormal contrasts such that intercepts represent the outcome distribution amongst "average" patients assuming equal weighting across all treatment groups.

The posterior distribution of each contrast of interest will be reported along with posterior summaries: median, 95\% highest density credible interval, and posterior probability of events of interest (e.g. that $\beta < 0$). Posterior summaries of other model parameters will be reported.


\subsection{Intercurrent Events}

For most analyses, the only intercurrent event (ICE) of concern is death. In most cases, this ICE is handled by the use of composite outcomes where death is assigned the worst possible score of the outcome on the ordinal scale.

For other intercurrent events (e.g. treatment switching/withdrawal) the treatment-policy strategy will be used with treatment groups analysed as assigned irrespective of withdrawal from or discontinuation of the assigned treatment.

\subsection{Primary Outcome}\label{sec:primary-outcome-analysis}
The clinical health status at day 14 will be summarised by counts and proportions of patients within each outcome level by study arm (i.e. the distribution of outcome levels). The following Bayesian cumulative logistic regression model will be updated using the available data.
$$
    \begin{aligned}
        y_{i} | \pi;x_i                            & \sim \text{Categorical}(\pi(x_i))                              \\
        \mathbb P(y_i \leq k | \alpha, \beta; x_i) & = \text{logit}^{-1}(\eta_{ik})                                 \\
        \eta_{i}                                   & = x_i^{\mathsf{T}}\beta + \cdots\quad(\text{other covariates}) \\
        \pi_k(\eta)                                & = \begin{cases}
            \text{logit}^{-1}(\alpha_1 - \eta)                                            & k=1             \\
            \text{logit}^{-1}(\alpha_{k} - \eta) - \text{logit}^{-1}(\alpha_{k-1} - \eta) & k\in\{2,...,7\} \\
            1 - \text{logit}^{-1}(\alpha_{7} - \eta)                                      & k=8
        \end{cases},
    \end{aligned}
$$
with prior distributions
$$
    \begin{aligned}
        \pi(0)          & \sim \text{Dirichlet}(\kappa)                            \\
        \beta_1,\beta_2 & \sim \text{Normal}(0, 1)                                 \\
        \kappa          & = 8\cdot(0.80, 0.11, 0.02, 0.02, 0.01, 0.01, 0.01, 0.02)
    \end{aligned}
$$
where the hyper-parameters $\kappa$ are weakly informed by the observed distribution at day 14 in CLARITY 1.

Due to the orthornormal design coding, the prior specified on $\pi(0)$ is the prior distribution on the outcome levels for an average patient with equal weighting given to each study arm, rather than the distribution of outcome levels for patients in the double placebo control group.

\subsection{Secondary Outcomes}

\subsubsection{Clinical health score at day 28}
The analysis will be analagous to that for clinical health score at day 14 (\ref{sec:primary-outcome-analysis}), but with a different prior. The prior is weakly informed by CLARITY 1 outcome data at day 28. For this outcome we specify
$$
    \kappa = (0.91, 0.02, 0.01, 0.01, 0.01, 0.01, 0.01, 0.02)
$$
for the hyper-parameter on the prior for $\pi(0)$.

\subsubsection{ICU admission (incidence days 0 - 28)}\label{sec:icu-analysis}

\textbf{Participants who die without ICU admission? Just $\mathbf{y_i=0}$? Or define as ordinal outcome with 0 - no ICU, 1 - ICU, 2 - Death? Or if die without ICU, count as ICU in any case, i.e. $\mathbf{y_i=1}$?}

The number and proportion of patients with any ICU admission in days 0 to 28 will be reported.The following Bayesian logistic regression model will be updated using the available data.
$$
    \begin{aligned}
        y_i|\pi_i & \sim \text{Bernoulli}(\pi_i)   \\
        \pi_i     & = \text{logit}^{-1}(\eta_i)    \\
        \eta_i    & = \alpha + x_i^\mathsf{T}\beta
    \end{aligned}
$$
with priors
$$
    \begin{aligned}
        \alpha & \sim \text{Logistic}(\text{logit}(0.2), 1)     \\
        \beta  & \overset{\text{iid}}{\sim} \text{Normal}(0, 1)
    \end{aligned}
$$
where the prior for $\alpha$ is weakly informed by CLARITY 1 results.

\subsubsection{Death (incidence days 0 - 28)}
This outcome will be analysed analgously to the ICU incidence outcome (\ref{sec:icu-analysis}). However, the following priors will instead be specified
$$
    \alpha \sim \text{Logistic}(\text{logit}(0.1), 1),
$$
where the prior is weakly informed by CLARITY 1 results.

\subsubsection{Time to death (from randomisation)}
Time to death will be censored at either the end of follow-up (\textbf{Protocol doesn't mention this explicitly, but I'm assuming death will be identified up to 180 days post-randomisation}) or in the case of loss-to-follow-up, at the last study day on which the participant was known to be alive.

Kaplan-Meier curves for time to death will be presented by treatment group. Time to death will be analysed a proportional hazards model. The baseline hazard will be modelled flexibly using M-splines.

Let $D_i = (y_i, \nu_i)$ denote the data for subject $i$ where $y_i$ is the event time and $\nu_i$ and indicator for censoring or observation of the event. The model for the hazard is
$$
    \begin{aligned}
        \lambda(t|\beta;x) & = \lambda_0(t)\exp(x^\mathsf{T}\beta)                           \\
        \lambda_0(t)       & = \sum_{l=1}^L M_l(t;\tau)\gamma_l,\quad \sum_{l=1}^L\gamma_l=1 \\
        \eta_i             & = x_i^\mathsf{T}\beta                                           \\
        \lambda_i(y_i)     & = \lambda_0(y_i)\exp(\eta_i)                                    \\
    \end{aligned}
$$
where $M_l(t;\tau)$ is the $l$th basis term for a M-spline of degree 3 with knots at locations $\tau=\{\tau_1,...,\tau_{L+4}\}$ evaluated at $t$. The default not location will be at the quantiles of observed event times. If this is determined to be inappropriate (too few events, insufficient flexible etc.) then an alternative may be specified.

The contrasts of interest will be the hazard ratio of the intervention group relative to each of the control groups.

If there is clear departure from proportional hazards assumption, then the treatment effect parameters $\beta$ may be allowed to vary with time rather than a time-invariant effect. In this case, restricted mean survival time will be the comparison of interest.

\subsubsection{Acute kidney injury (incidence in days 0 - 28)}
This outcome will be analysed analgously to the ICU incidence outcome (\ref{sec:icu-analysis}). However, the following priors will instead be specified
$$
    \alpha \sim \text{Logistic}(\text{logit}(0.2), 1),
$$
where the prior is weakly informed by CLARITY 1 results.

\subsubsection{Respiratory failure (incidence in days 0 - 28)}
This outcome will be analysed analgously to the ICU incidence outcome (\ref{sec:icu-analysis}). However, the following priors will instead be specified
$$
    \alpha \sim \text{Logistic}(\text{logit}(0.2), 1),
$$
where the prior is weakly informed by CLARITY 1 results.

\subsubsection{Length of hospital admission (days of inpatient stay from admission to discharge or death)}

\textbf{Death as a competing risk for time-to-discharge. Multi-state model hospitalised -> discharge -> death, hospitalised -> death. Protocol doesn't specify censoring time for this, again assuming 180 days (last day of follow-up).}

A multi-state hazard model...

\subsubsection{Length of ICU admission (days in ICU from admission to transfer to ward or death)}

\textbf{Either just ordinal, or similarly as for length of hospital admission including new state: hospital -> icu -> discharge, hospital -> icu -> death, hospital -> icu -> hospital -> discharge etc.}

\subsubsection{Requirement of ventilatory support (number of days with ventilation in days 0 - 28)}\label{sec:vent-support}

\textbf{Code deaths as worst outcome, i.e. 28 days? What about people who die without requiring any dialysis? Are they the same as someone dying and requiring 28 days with dialysis?}

Counts and distributions of the outcome will be summarised by treatment group. A cumulative logistic proportional odds model will be assumed. If $y_i\in\{0,1,...,28,29\}$ denotes the number of days with ventilatory support (where 0 means 0 days, and 29 corresponds with death) for participant $i$, and $x_i$ the intervention design covariates, then
$$
    \begin{aligned}
        y_{i} | \pi;x_i                            & \sim \text{Categorical}(\pi(x_i))           \\
        \mathbb P(y_i \leq k | \alpha, \beta; x_i) & = \text{logit}^{-1}(\eta_{ik})              \\
        \eta_{ik}                                  & = \alpha_k + x_i^{\mathsf{T}}\beta + \cdots \\
        \pi_k(x)                                   & = \begin{cases}
            1 - \text{logit}^{-1}(\alpha_1 + x^\mathsf{T}\beta)                                                     & k=0               \\
            \text{logit}^{-1}(\alpha_{k-1} + x^\mathsf{T}\beta) - \text{logit}^{-1}(\alpha_{k} + x^\mathsf{T}\beta) & k\in\{2,...,K-1\} \\
            \text{logit}^{-1}(\alpha_{K-1} + x^\mathsf{T}\beta)                                                     & k=29
        \end{cases},
    \end{aligned}
$$
with prior distributions
$$
    \begin{aligned}
        \pi(0)          & \sim \text{Dirichlet}(\kappa) \quad \text{(weakly informed by CLARITY 1)} \\
        \beta_1,\beta_2 & \sim \text{Normal}(0, 1).
    \end{aligned}
$$
The primary contrasts will be the relative log-odds of having a worse outcome (higher outcome level) in the investigational arm compared to each of the contorl arms.

Outcome levels with at least 1 observation will be included in the model. If any outcome levels have 0 observations then these will be excluded from the model.

Non-proportionality, particularly with respect to death, to be investiated.

\subsubsection{Requirement of dialysis (number of days with dialysis in days 0 - 28)}

This outcome will be analysed analgously to \ref{sec:vent-support} but with prior hyper-parameter
$$
    \kappa = \cdot \quad \text{(weakly informed by CLARITY 1)}
$$
for $\pi(0)$.

\subsubsection{Clinical health score at day 60}
The analysis will be analagous to that for clinical health score at day 14 (\ref{sec:primary-outcome-analysis}), but with a different prior. The prior is weakly informed by CLARITY 1 outcome data. For this outcome we specify
$$
    \kappa = \cdots
$$
for the hyper-parameter on the prior for $\pi(0)$.

\subsubsection{Clinical health score at day 90}
The analysis will be analagous to that for clinical health score at day 14 (\ref{sec:primary-outcome-analysis}), but with a different prior. The prior is weakly informed by CLARITY 1 outcome data. For this outcome we specify
$$
    \kappa = \cdots
$$
for the hyper-parameter on the prior for $\pi(0)$.

\subsubsection{Clinical health score at day 180}
The analysis will be analagous to that for clinical health score at day 14 (\ref{sec:primary-outcome-analysis}), but with a different prior. The prior is weakly informed by CLARITY 1 outcome data. For this outcome we specify
$$
    \kappa = \cdots
$$
for the hyper-parameter on the prior for $\pi(0)$.

\subsection{Safety Outcomes}

For safety outcomes the number (and proportion) of participants experiencing the event of interest (hypotension, hyperkalaemia, deranged liver function, SAEs) and the total number (and rate) of events will be reported by treatment arm.


\subsection{Baseline Adjustments}
As randomisation is stratified by centre, all models will include centre as a random effect. The effect of centre will be assumed additive in the linear predictor and will be modelled by
$$
    \begin{aligned}
        \gamma_j|\tau & \sim \text{Normal}(0, \tau^2), \quad j =1,...,J \\
        \tau          & \sim \text{Cauchy}(0, 2.5).
    \end{aligned}
$$

\subsection{Subgroup Analyses}
No pre-specified subgroup analyses are planned.

\subsection{Missing Data}
Patterns of missing baseline and outcome data will be reported by treatment group.

Where other sources of information are available, missing baseline variables may be imputed deterministically. Missing outcomes may be treated as censored if appropriate (e.g. if a participant was known to be alive at day 14, but their exact status was unknown then their outcome is interval censored in $\{1,...,7\}$) or if they were known to be out of hospital but with unknown not-hospitalised status then their outcome is interval censored in $\{1,2\}$).

Baseline predictors of missingness will be investigated. In the absence of strong predictors of missingness, the default analysis will be based on available cases adjusting for the pre-specified covariates. Alternatively, missing outcomes may be multiply imputed using an expanded set of baseline covariates and the combined results used for reporting.

\subsection{Software}
The statistical software R will be used for data processing. Bayesian model posteriors will be approximated using HMC as implemented in Stan. Program and package versions used for the analyses will be reported.

\clearpage

\section{Trial Monitoring and Reporting}

Analyses are planned to occur at the following stages of recruitment:

\begin{itemize}
    \item Exploratory safety analysis after 80 particiants randomsied 1:1 to two arms.
    \item Effectiveness analysis after an additional 520 participants randomised 1:1:1 to all 3 arms.
    \item Effectiveness analysis after every additional 600 participants up to the maximum sample size of 2,100 participants.
\end{itemize}

At each analysis, the investigational arm will be compared with both control arms with respect to the primary outcome.

\subsection{Decision Rules}

Decisions pertaining to continuation of the trial will be guided by the calculation of predictive probabilities of satisfying the pre-specified criteria of effectiveness with respect to the primary outcome.

The quantities of interest are the predictive probability of trial success for each of the contrasts, defined as
\begin{align}
    \text{PPoS}_{\text{P+P}}(\texttt{data}_n, m) & = \mathbb{E}[\mathbb{P}(\eta_3 - \eta_2 > \delta | \texttt{data}_{n+m}) > \epsilon_{\text{eff}} | \texttt{data}_n] \\
    \text{PPoS}_{\text{C+P}}(\texttt{data}_n, m) & = \mathbb{E}[\mathbb{P}(\eta_3 - \eta_1 > \delta | \texttt{data}_{n+m}) > \epsilon_{\text{eff}} | \texttt{data}_n]
\end{align}
where (1) relates to the comparison of the investigational arm with control arm 1 and (2) the comparison of the investigational arm with control arm 2.

Given the aim of the trial is to establish effectiveness relative to both of the control arms, if either of the quantities falls below a threshold, $\epsilon_{\text{fut}}$, at an interim analysis then a decision of trial futility is recommended.

Calculation of these predictive quantities requires assumptions about the future distribution of covariates included in the primary model. The assumption will be that future participants have similar covariates to past participants.

\clearpage

\printbibliography[heading=bibintoc]

\clearpage

\section*{Appendix I}

\end{document}
